\documentclass[letterpaper, 10pt]{article}
\usepackage[margin=1.25in]{geometry}

% Title Page
\title{TSA Software Development \\ Regionals - Flower Mound, Texas}
\author{Sammy Shin, Trevor Nguyen, Christian Duffee}
\date{March 2016}

\begin{document}
\maketitle

\cleardoublepage

\tableofcontents

\cleardoublepage
\section{Research}	
	Several papers have already been published about artificial neural networks and work has been done almost all major universities. A significant portion or our inspiration has come from the work of the University of Klagenfurt in Austria. In a paper published by its Institute for Networked and Embedded Systems, we have examined the details of how neural networks worked abstractly and we knew the major concepts and general algorithms to implement. Further research went into discovering the art of designing genetic algorithms in order to model the entities as close to real life as possibly by emulating the way the evolution happens in populations over time yet allowing this process to be seen in real time. To view the genetic evolutionary process would be very slow as this process happens over several generations in real life, so what we have created uses generations of very short time spans and rapid mutations which would be most akin to observing a small population of bacteria. The California Institute of Technology also has an on-line lecture series covering advanced artificial intelligence projects which we also drew inspiration to create our own rendition on neural networks.
	
	
	\paragraph{Backward Propagation}
	Backward propagation is how the neural network is able to improve itself over time. It includes comparing the desired outputs with the network's outputs.
\cleardoublepage
\section{Description}
	\paragraph{Problem} A large problem in computation lies in humans thinking of, designing, and implementing algorithms in software. Algorithm design for extremely esoteric scenarios can be time-consuming. For example, to design an algorithm to make individual decisions for a distributed system of robots, would require deep analysis of emergent behaviors as the scale becomes smaller, as well as attention to the larger aspect of things with large-scale integration and cooperation between individually functioning entities.
	\paragraph{Solution} The project aims to create a neural network toolkit as well as a demonstrative displaying simulations of genetic algorithms visually in order to convey its usefulness. The demonstration will be displaying creatures in a simple environmental scenario of predation and survival.
	\paragraph{Value} In an epic test of survival and a strong application of Charles Darwin's theories to explain evolution and the concept of natural selection, the software comes with toolkits in order to determine outputs from various inputs. The demonstration has educational value of teaching young people the prominence of computation in solving problems in everyday life as well as in industrial applications. Although not directly shown in our demonstration, the technology of neural networks can be applied to fields in image recognition and attempts at replicating human creativity. Deep learning is quite feasibly the future role of computer science.
\cleardoublepage
\section{TSA Plan of Work Log}
	
	Advisor signature:
	\rule{10cm}{0.4pt}
	\footnotesize
	\begin{center}
	\begin{tabular} { l || p{1.25 in}| l| l|p{2.75in}}
		Date & Task & Time & Team member & Comments \\
		03/04/16 & Plan abstract layouts & 50 min & Sammy Shin & This process took longer than expected. \\
		03/04/16 & Java class design & 45 min & Trevor Nguyen & Designing object oriented fields and methods.\\
		03/04/16 & Begin coding framework & 240 min & Sammy Shin & Good progression in the development cycle.\\
		03/04/16 & Finish main project framework & 30 min & Trevor Nguyen & Project is not complete yet.\\
		03/04/16 & Documentation process & 100 min & Sammy Shin & Documentation was quicker than projected.\\
		03/04/16 & Create demonstration program & 125 min & Christian Duffee & The graphics programming caused some complications.\\
		03/04/16 & Finalize design and make final revisions & 45 min & (all) & Debugging was easy with consistent code.\\
		
	\normalsize
	\end{tabular}
	\end{center}
\cleardoublepage
\section{Software Design Process}
	\subsection{Project Requirements}
		\paragraph{Usability} We knew that our software had to be usable by other programmers and software engineers. In order to fulfill this requirement, we made our code extremely logical without sacrificing our reputation for creating efficient algorithms. In addition, we were able 
	\subsection{High-level Software Design}
		\paragraph{Artificial Neural Networks}
		The project is entirely based around artificial neural networks that add a layer of abstraction to our design process. This concept is a form of deep learning that is directly modeled off of the biological neural networks that are present in almost all Phyla of the Kingdom Animalia.
		\paragraph{Abstract Algorithms}
		In order to discover the actual algorithms to govern the organisms would be too difficult to make them both efficient yet seem realistic. This is where artificial neural networks become useful because they allow for specific algorithms to be discovered randomly by a computational process.
	\subsection{Testing}
		In order to test the development toolkit, we created a quick graphical demonstration of how the software can be used in an actual application. Although the demonstration isn't completely technical, it still accurately communicates the vast capabilities of our software. A lot of debugging went into the project very nigh to the time of shipping the product. This was because in order to create the demonstration, we were required to push ourselves beyond our comfort zone in the field of video game design. We were required to implement graphical displays using the built-in Java swing graphics package in order to display information to the screen.
		\paragraph{Demonstration} In all simplicity, our program was designed as a neural network toolkit for other developers to implement in their respective projects and ideas. We have demonstrated what a possible use could be for our toolkit by creating an environment in which many organisms (signified by the cyan colored blocks) dubbed as Malishes which is a rough Russian translation for small creature. The purpose of these creatures is to gather food (which is denoted by the green blocks) and reproduce. The longer a Malish goes without eating food, it becomes hungrier and dies. This hunger is visible through the creature changing its color from cyan to yellow, for slightly hungry, to red, about to die from hunger. Our neural network is seen in these creatures through their adaptive behavior to eat food more efficiently.  
	\subsection{End User Product Documentation}
		In order to make the end user experience as great as possible, we used Java documentation style as much as humanly possible in order to make the software toolkit extremely usable. From our diligently commented and annotated code, anyone who is familiar with the inner-workings of Java and experience using something like JavaDocs will be able to use the software in anything from hobby projects to industrial applications.
\section{Team Evaluation and Future Prospects}
	Our team strongly believes in the project and it has been a great time working with it to make the program function as intended. Despite our successes, we believe that it would have been even more successful if the time line of the project had been spread out over a longer period of time to allow ideas to settle and problems to be resolved with much more deliberation.
\cleardoublepage
\section{References and Further Readings}
	MIT OpenCourseWare Artificial Intelligence 2006 Lecture Series \\ \\
	Evolving Neural Network Controllers for a Team of
	Self-organizing Robots\\\\
	Washington University: Introduction to Neural Networks
\end{document}          
 