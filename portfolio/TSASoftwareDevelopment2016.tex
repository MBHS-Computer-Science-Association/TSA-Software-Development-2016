\documentclass[letterpaper, 10pt]{article}
\usepackage[margin=1.0in]{geometry}
\usepackage{graphicx}
\usepackage{verbatim}

\begin{document}

% Title Page
\begin{titlepage}
	\centering	
	\includegraphics[width=6.50cm]{res/TSA-Emblem.png}
	\vspace{1.15cm}
	
	\scshape\Huge TSA Software Development \\
	\vspace{0.25cm}
	\scshape\huge Texas State Conference -- Waco, Texas \\
	\vspace{0.25cm}
	\scshape\LARGE April 11th - 16th, 2016 \\
	
	\vspace{1.55cm}
	
	\mdseries\Large McKinney Boyd High School Chapter \\
	\vspace{0.25cm}
	\slshape\LARGE Sammy Shin, Trevor Nguyen, Christian Duffee \\
	\vspace{0.25cm}
	\scshape\LARGE Spring 2016 \\
\end{titlepage}

\cleardoublepage

\Large
\tableofcontents
\normalsize

\cleardoublepage

\section{Research}
	We focused our research on the topic of artificial neural networks (ANNs), which although relatively new grounds for our project team, ANNs is not a novel topic to academia. Several academic papers in computer science have been long since published about using ANNs in a variety of scenarios from analyzing partial derivatives to recognizing visual data. With a abstract knowledge of ANNs solved problems with software, we delved into the capabilities and limits of what ANNs could do in order to establish a deep understanding of their context in applications. We then spent the rest of our research on current implementation techniques which unsurprisingly required knowledge of mathematics beyond the scope of high school calculus. As difficult as it was rewarding, our research has culminated in a compilation of our findings below.
	\paragraph{The Artificial Neural Network}
		 The term \textbf{artificial neural network} refers to a data model which provides means to compute a set of outputs from a set of inputs which is specifically modeled after the \textbf{biological neural networks} present in animals. Biological neural pathways are based around impulses causing neurons to selectively fire after reaching a threshold. Whether a neuron will fire depends on multiple inputs of varying importance. When an active neuron fires, it stimulates the firing of other neurons further along the pathway. Computer scientists have likened this biological relationship to \textbf{weighted digraphs}, in which nodes represent neurons and are connected by unidirectional edges that have numerical weights associated with them. In ANNs, inputs are fed into one side of the digraph, and after series of abstract computations called \textbf{transfer functions}, outputs emerge on the other side. ANNs are, by nature, highly abstracted data models.
	\paragraph{Capability}
	ANNs are very good at pattern recognition because they are modeled off of the universe's most sophisticated decision making machine, the \emph{brain}. They have been able solve some of the most difficult problems with orders of magnitude more accurate than more traditional algorithms. ANNs are used in \textbf{natural language processing}, \textbf{image identification}, \textbf{speech recognition}, \textbf{handwriting recognition}, and recently, playing Go. Despite their robustness, ANNs are limited by their need for extremely large data sets from which they extrapolate algorithms.
	\paragraph{Transfer Functions}
	When all of a node's input values are available for computation, the node runs what is called a transfer function, which will calculate that specific node's output value. The transfer function takes a set of inputs and a set of corresponding weights as formal parameters and processes them with higher weighted inputs having more of an effect on the node's output. In addition to inputs, the transfer function also takes in a \textbf{threshold value} which will must be reached in order to have a signal propagate.
	\paragraph{Backward Propagation}
	Backward propagation is how ANNs adjust graph weights in order to improve network accuracy and reduce \textbf{systematic errors}. The network's compares its outputs with the desired outputs in order to calculate that error. The output nodes propagate the error to their respective input nodes and the individual input nodes' error is calculated mathematically.
	\paragraph{Concurrency}
	When a program runs on more than one thread and does not conform to a simple processing in sequential order, the program exhibits \textbf{concurrency}. For ANNs, the transfer functions of each node run independently of each other and instead of processed sequentially, they are often processed in parallel.
	\paragraph{Parallel Computing and Process Synchronization}
	Both of these concepts crop up in the world of concurrency and are techniques in order to effectively use a programming language's concurrency capabilities to one's advantage. \textbf{Parallel computing} is the act of running computational processes at the same time on different program threads. The actual hardware processor may use its own algorithms to handle these low level task, which is normally going to be some derivative of round-robin scheduling. With many independent asynchronous processes running in parallel, the issue of \textbf{process synchronization} emerges, which is essentially the unification of multiple processes under the same time domain. Classic problems such as the producer-consumer problem emerge in concurrent applications and desperately need synchronization in order to be stable.
	
\cleardoublepage
\section{Description}
	\paragraph{Problem} The job of a software engineer is to maintain software code and to write algorithms to perform necessary operations on data while keeping in mind clarity, correctness, and efficiency. Being a software engineer is an example of a difficult job in high demand, requiring a very specific mindset and vast intellectual capability. Software engineers handle tasks at which ordinary people without the proper wherewithal would balk.
	
	Algorithms dominate the software engineering sphere. A large problem in applied computer science lies in humans thinking of, designing, and implementing algorithms in real software that can be run efficiently given hardware constraints. There is a large list of well-known algorithms that almost all computer scientists understand and study as part of their computer science degree, yet this list of algorithms that are studied is finite their immediate applications are limited. Known algorithms are efficient and used where possible, but when a problem requires an algorithm that has not yet been created, these well-known algorithms can only serve as a computer scientist's inspiration.
	
	Algorithm design for extremely esoteric scenarios can be time-consuming for software engineers, backing up the software design process, creating delays, and losing profits for a company. In short, algorithm design would be the major bottleneck for any sort of large software project. Too often, a company may sacrifice efficiency for speed of development, resulting in less-than-optimized algorithms. The larger problem, however is coming up with algorithms for which it would be completely infeasible to design by hand.
	
	For example, if one were to design software for a distributed system of robots, one would have to write the decision making model for each one of these robots. However, one must keep in mind the greater scale of the system complete with group interactions and changing environments. Deep analysis is required to determining emergent behaviors as the scale becomes smaller, as well as attention to the larger aspect of things with large-scale integration and cooperation between individually functioning entities. To put things into perspective, in order to organize a mob with a size on the order of one thousand people, it would require a massive organization
	\paragraph{Solution} The project aims to create a neural network toolkit as well as a demonstrative displaying simulations of genetic algorithms visually in order to convey its usefulness. The demonstration will be displaying creatures in a simple environmental scenario of predation and survival.
\clearpage
	\paragraph{Value} In an epic test of survival and a strong application of Charles Darwin's theories to explain evolution and the concept of natural selection, the software comes with toolkits in order to determine outputs from various inputs. The demonstration has educational value of teaching young people the prominence of computation in solving problems in everyday life as well as in industrial applications. Although not directly shown in our demonstration, the technology of neural networks can be applied to fields in image recognition and attempts at replicating human creativity. Deep learning is quite feasibly the future role of computer science.
\cleardoublepage
\section{TSA Plan of Work Log}
	
	Advisor signature:
	\rule{10cm}{0.4pt}
	\footnotesize
	\begin{center}
	\renewcommand{\arraystretch}{2.15}
	\begin{tabular} { l || p{1.25 in}| l| l|p{2.50in}}
		Date 	& Task 								& Time 	& Initials 		& Comments \\
		02/14/16 	& Plan abstract layouts 					& 50 min 	& SS, TN, CD	& This process took longer than expected. \\
		02/23/16 	& Java class design 						& 45 min 	& TN, CJ 		& Designing object oriented fields and methods.\\
		02/24/16 	& Begin coding framework 				& 240 min & SS, TN, CD	& Good progression in the development cycle.\\
		02/26/16 	& Finish main project framework 			& 30 min	& TN, CJ		& Project is not complete yet.\\
		03/01/16 	& Documentation process 				& 100 min	& SS, TN 		& Documentation was quicker than projected.\\
		03/01/16 	& Create demonstration program 			& 125 min	& CD, SS 		& The graphics programming caused some complications.\\
		03/03/16 	& Finalize design and make final revisions 	& 45 min 	& SS, TN, CD 	& Debugging was easy with consistent code.\\
		03/08/16 	& 									& 55 min 	& TN, CJ 		& \\
		03/09/16 	& 					 				& 120 min & SS, TN, CD	& \\
		03/12/16 	&  									& 30 min	& TN, CJ		& \\
		03/13/16 	&  									& 135 min	& SS, TN 		& \\
		03/15/16 	&  									& 15 min 	& TN, CJ 		& \\
		03/18/16 	&  									& 25 min 	& SS, TN, CD	& \\
		03/19/16 	&  									& 180 min	& TN, CJ		& \\
		03/23/16 	&  									& 135 min	& SS, TN 		& \\
		03/25/16 	&  									& 55 min	& SS, TN 		& \\
		03/26/16 	&  									& 80 min 	& TN, CJ 		& \\
		03/29/16 	&  									& 25 min 	& SS, TN, CD	& \\
		04/02/16 	&  									& 170 min	& TN, CJ		& \\
		04/02/16 	& 					 				& 105 min	& SS, TN 		& \\
		
	\end{tabular}
	\end{center}
	
	\normalsize
	
\cleardoublepage
\section{Software Design Process}
	\subsection{Project Requirements}
		\paragraph{Usability} We knew that our software had to be usable by other programmers and software engineers. In order to fulfill this requirement, we made our code extremely logical without sacrificing our reputation for creating efficient algorithms. In addition, we were able 
		
		
	\subsection{High-level Software Design}
		\paragraph{Artificial Neural Networks}
		The project is entirely based around artificial neural networks that add a layer of abstraction to our design process. This concept is a form of deep learning that is directly modeled off of the biological neural networks that are present in almost all Phyla of the Kingdom Animalia.
		\paragraph{Abstract Algorithms}
		In order to discover the actual algorithms to govern the organisms would be too difficult to make them both efficient yet seem realistic. This is where artificial neural networks become useful because they allow for specific algorithms to be discovered randomly by a computational process.
		\paragraph{Weakly Referenced Nodes}
		\paragraph{Optimized Graph Data Model}
		\paragraph{Asynchronous Transfer Functions}
	\subsection{Testing}
		In order to test the development toolkit, we created a quick graphical demonstration of how the software can be used in an actual application. Although the demonstration isn't completely technical, it still accurately communicates the vast capabilities of our software. A lot of debugging went into the project very nigh to the time of shipping the product. This was because in order to create the demonstration, we were required to push ourselves beyond our comfort zone in the field of video game design. We were required to implement graphical displays using the built-in Java swing graphics package in order to display information to the screen.
		\paragraph{Demonstration} In all simplicity, our program was designed as a neural network toolkit for other developers to implement in their respective projects and ideas. We have demonstrated what a possible use could be for our toolkit by creating an environment in which many organisms (signified by the cyan colored blocks) dubbed as Malishes which is a rough Russian translation for small creature. The purpose of these creatures is to gather food (which is denoted by the green blocks) and reproduce. The longer a Malish goes without eating food, it becomes hungrier and dies. This hunger is visible through the creature changing its color from cyan to yellow, for slightly hungry, to red, about to die from hunger. Our neural network is seen in these creatures through their adaptive behavior to eat food more efficiently.  
	\subsection{End User Product Documentation}
		In order to make the end user experience as great as possible, we used Java documentation style as much as humanly possible in order to make the software toolkit extremely usable. From our diligently commented and annotated code, anyone who is familiar with the inner-workings of Java and experience using something like JavaDocs will be able to use the software in anything from hobby projects to industrial applications.
\section{Team Evaluation and Future Prospects}
	Our team strongly believes in the project and it has been a great time working with it to make the program function as intended. Despite our successes, we believe that it would have been even more successful if the time line of the project had been spread out over a longer period of time to allow ideas to settle and problems to be resolved with much more deliberation.
\cleardoublepage
\section{References and Further Readings}
	We encountered many extremely invaluable resources during our team's research which allowed us to attain a well-rounded, responsible working knowledge of artificial neural networks by the time we were finished with the project.

	\large
	\vspace{0.65cm}

	\begin{itemize}
		\item MIT OpenCourseWare 6.034 Artificial Intelligence
		
			\hspace{1.5cm} Fall 2010 Video Lecture Series
		\item Evolving Neural Network Controllers for a Team of Self-organizing Robots
		
			\hspace{1.5cm} University of Klagenfurt, Klagenfurt, Austria -- Istav\'{a}n Feh\'{e}rv\'{a}ri
		\item Caltech Machine Learning Course -- CS 156
		
			\hspace{1.5cm} Spring 2012 Video Lecture Series
		\item Artificial Intelligence: A Modern Approach Textbook
			
			\hspace{1.5cm} Stuart Russell and Peter Norvig
		\item The Java Language Specification
			
			\hspace{1.5cm} Java SE 8 Edition, Oracle
	\end{itemize}
\end{document}          
 